% !TeX root = RJwrapper.tex
\title{R Packages to Aid in Handling Web Access Logs}
\author{by Oliver Keyes, Bob Rudis}

\maketitle

\abstract{
Web access logs contain information on HTTP(S) requests and form a key
part of both industry and academic explorations of human behaviour on
the internet - explorations commonly performed in R. In this paper we
explain and demonstrate a series of packages - \emph{webreadr},
\emph{urltools}, \emph{iptools} and \emph{rgeolocate} - designed to
efficiently read in, parse and munge access log data, allowing
researchers to handle it easily.
}

\subsection{Introduction}\label{introduction}

Introductory section which may include references in parentheses
\citep{R}, or cite a reference such as \citet{R} in the text.

\subsection{Section title in sentence
case}\label{section-title-in-sentence-case}

This section may contain a figure such as Figure \ref{figure:rlogo}.

\begin{figure}[htbp]
  \centering
  \includegraphics{Rlogo}
  \caption{The logo of R.}
  \label{figure:rlogo}
\end{figure}

\subsection{Another section}\label{another-section}

There will likely be several sections, perhaps including code snippets,
such as:

\begin{Schunk}
\begin{Sinput}
x <- 1:10
x
\end{Sinput}
\begin{Soutput}
#>  [1]  1  2  3  4  5  6  7  8  9 10
\end{Soutput}
\end{Schunk}

\subsection{Summary}\label{summary}

This file is only a basic article template. For full details of
\emph{The R Journal} style and information on how to prepare your
article for submission, see the
\href{http://journal.r-project.org/latex/RJauthorguide.pdf}{Instructions
for Authors}.

\bibliography{RJreferences}

\address{
Oliver Keyes\\
Wikimedia Foundation\\
line 1\\ line 2\\
}
\href{mailto:author1@work}{\nolinkurl{author1@work}}

\address{
Bob Rudis\\
Rapid7\\
line 1\\ line 2\\
}
\href{mailto:author2@work}{\nolinkurl{author2@work}}

